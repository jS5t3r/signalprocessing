%% LyX 2.2.2 created this file.  For more info, see http://www.lyx.org/.
%% Do not edit unless you really know what you are doing.
\documentclass[twoside,english]{article}
\usepackage[T1]{fontenc}
\usepackage[latin9]{inputenc}
\usepackage{geometry}
\geometry{verbose,lmargin=2cm,rmargin=2cm}
\usepackage{fancyhdr}
\pagestyle{fancy}
\setlength{\parindent}{0bp}
\usepackage{babel}
\usepackage{float}
\usepackage{amsmath}
\usepackage[unicode=true,
 bookmarks=true,bookmarksnumbered=false,bookmarksopen=false,
 breaklinks=false,pdfborder={0 0 1},backref=false,colorlinks=false]
 {hyperref}
\hypersetup{
 pdfauthor={Peter Lorenz}}

\makeatletter
%%%%%%%%%%%%%%%%%%%%%%%%%%%%%% User specified LaTeX commands.
\date{}

\usepackage{todonotes}
\bibliographystyle{plain}
\usepackage{fancyhdr}
\fancyhf{}% Clear header footer

\makeatother

\begin{document}

\lhead{LORENZ Peter, REICHEL Robert\hfill{} Assignment 3}

\section*{Analytische Aufgabe 3.1}

\section*{(a) }

\[
X(z)=\frac{(z-1)(z+\frac{3}{2})}{(z-\frac{1}{2})(z+\frac{j}{2})(z-\frac{j}{2})},\;H_{1}(z)=(z-\frac{j}{2})
\]

Aus der Formelsammlung haben wir folgenden Zusammenhang gefunden:

\[
Y_{1}(z)=X(z)H_{1}(z)\longleftrightarrow y_{1}[n]s=(x*h_{1})[n]
\]

Wenn wir den Z�hler erweitern mit $H_{1}(z)=(z-\frac{j}{2})$:
\[
Y_{1}(z)=\frac{(z-1)(z+\frac{3}{2})(z-\frac{j}{2})}{(z-\frac{1}{2})(z+\frac{j}{2})(z-\frac{j}{2})}=\frac{(z-1)(z+\frac{3}{2})}{(z-\frac{1}{2})(z+\frac{j}{2})}
\]

Im Z�hler kann man nun die Nullstellen ablesen $NST_{1}=1$ und $NST_{2}=-\frac{3}{2}$.

\begin{figure}[H]

\caption{Da $h_{i}$ und $x[n]$ kausal sind, ist das ROC das einer kausalen
Folge.}

\end{figure}

$y_{1}[n]$ kann nicht reellwertig sein. Eine rellewerte Funktion
besitzt ausschlie�lich komplex konjugierte Pol- \& Nullstellen. Bei
$y_{1}[z]$ tritt der Pol $P_{1}=-\frac{j}{2}$ nicht komplex konjugiert
als $P_{2}=\frac{j}{2}$ auf. Daher kann $y_{1}[n]$ nicht rein reellwertig
sein. 

\section*{(b)}

\[
Y_{2}=X(z)\cdot H_{2}(z)=\frac{(z-1)(z+\frac{3}{2})}{(z-\frac{1}{2})(z+\frac{j}{2})(z-\frac{j}{2})}\cdot\frac{1}{z}=X(z)\cdot z^{-1}
\]

Aus der Formelsammlung:

\[
x[n-M]\longleftrightarrow X(z)z^{-M}
\]

Daraus l�sst sich das Ergebnis mit einer Verschiebung um 1 darstellen: 

\[
y_{2}[n]=x[n-1]
\]


\section*{(c)}

\[
Y_{3}(z)=X(z)\cdot H_{3}(z)=\frac{(z-1)(z+\frac{3}{2})}{(z-\frac{1}{2})(z-\frac{j}{2})(z+\frac{j}{2})}
\]

\begin{figure}[H]
\caption{ROC von $Y_{3}(z)$}
\end{figure}

Ein System ist minimalphasig, wenn die z-Transformierte und dessen
Inverse BIBO stabil sind. Ein System ist BIBO stabil, wenn der Einheitskreis
im ROC liegt. Bei $Y_{3}(z)$ ist dies der Fall, bei $Y_{3}(z)^{-1}$
jedoch nicht, da die Polstell von $Y_{3}(2)^{-1}$ am Einheitskreis
liegt und das ROC somit au�erhalb des Einheitskreis leigt. Daraus
folgt, $Y_{3}(z)$ ist nicht minimalphasig!

\section*{(d)}

\[
Y_{4}(z)=X(z)\cdot H_{4}(z)=\frac{z+\frac{3}{2}}{(z+\frac{j}{2})(z-\frac{j}{2})(z-\frac{1}{2})}
\]

\[
Y_{4}(z)=\underset{X(z)}{\underbrace{\frac{(z+\frac{3}{2})(z-1)}{(z+\frac{1}{2})(z-\frac{j}{2})(z-\frac{1}{2})}}}\cdot\underset{H_{4}(z)}{\underbrace{\frac{1}{z-1}}}
\]

\begin{figure}[H]
\caption{ROC von $Y_{4}(z)$}
\end{figure}

$H_{4}$ist nicht stabil, da der Einheitskreis nicht im ROC liegt.
\end{document}
\grid
